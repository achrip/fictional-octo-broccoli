\documentclass[conference]{IEEEtran}
\IEEEoverridecommandlockouts
% The preceding line is only needed to identify funding in the first footnote. If that is unneeded, please comment it out.
% \usepackage{cite}
\usepackage[sorting=none,
backend=biber]{biblatex}
\usepackage{amsmath,amssymb,amsfonts}
\usepackage{algorithmic}
\usepackage{graphicx}
\usepackage{textcomp}
\usepackage{xcolor}
\def\BibTeX{{\rm B\kern-.05em{\sc i\kern-.025em b}\kern-.08em
T\kern-.1667em\lower.7ex\hbox{E}\kern-.125emX}}
\addbibresource{bibliography.bib}

\begin{document}

\title{RAG Approach for Indonesia NCVS Exploration}

\makeatletter
\newcommand{\linebreakand}{%
  \end{@IEEEauthorhalign}
  \hfill\mbox{}\par
  \mbox{}\hfill\begin{@IEEEauthorhalign}
}
\makeatother

\author{\IEEEauthorblockN{Ashraf Alif Adillah}
\IEEEauthorblockA{\textit{School of Computer Science} \\
\textit{BINUS University}\\
Jakarta, Indonesia\\
ashraf.adillah@binus.ac.id}
\and
\IEEEauthorblockN{Ivan Afrizal}
\IEEEauthorblockA{\textit{School of Computer Science} \\
\textit{BINUS University}\\
Jakarta, Indonesia\\
ivan.afrizal@binus.ac.id}
\linebreakand
\IEEEauthorblockN{Meiliana}
\IEEEauthorblockA{\textit{School of Computer Science} \\
\textit{BINUS University}\\
Jakarta, Indonesia\\
meiliana@binus.edu}
\and
\IEEEauthorblockN{Alfi Yusrotis Zakiyyah}
\IEEEauthorblockA{\textit{School of Computer Science} \\
\textit{BINUS University}\\
Jakarta, Indonesia\\
alfi.zakiyyah@binus.edu}
}

\maketitle

\begin{abstract}
\end{abstract}

\begin{IEEEkeywords}
\end{IEEEkeywords}

\section{Introduction}
In recent years, Language Models (LMs) have been developing in such an
astonishing rate; what once was a basic sentiment analysis algorithm is now able to handle
increasingly complex tasks such as long form question answering \cite{stelmakh-etal-2022-asqa,fan-etal-2019-eli5},
open-domain sumarization \cite{cohen-etal-2021-wikisum,giorgi-etal-2023-open,hayashi-etal-2021-wikiasp}
and chain-of-thought (CoT) reasoning \cite{geva-etal-2021-aristotle,hendrycks-etal-2020-measuring,
ho-etal-2020-constructing,wei-etal-2022-chain,geva-etal-2021-aristotle}. This remarkable progress has
led to a wide range of applications across various domains, including the legal and law sector, where
language models are now utilized for ... \cite{} % harus ada yang menjelaskan tentang legal law (apa contohnya)

% Tambahin buat bahas BPK itu apa yang kemungkinan adalah fungsi bpk ah... 
While Indonesia's BPK portal \cite{} provide a comprehensive database of the country's laws, it is not without
its limitations. For example, the sheer volume of legal information can be overwhelming, and navigating through
the website can be cumbersome. However, there is significant potential in applying LMs to improve the current
situation. Through their capabilities, it may be possible to develop more effective search algorithms, enhance
the accessibility of legal information, even automate the process of legal research, which would ultimately
make it easier for citizens and legal professionals alike to navigate and utilize the vast repository of
Indonesian laws.

In this paper, we propose a generative model that is able to respond to a certain statutory law questions.
Namely, Indonesia's Non-Convention Vessel Statutories \cite{}. By leveraging a retriever that works over the corpus
of statutories to fetch a set of corresponding legislative articles, the generative model would then be able
to formulate a more comprehensive answer. This popular methodology known as the "retrieve-then-read" pipeline
\cite{} is the foundation for our proposed solution to the problem stated above.

% Final remark 1: Secara keseluruhan oke, cuma kalau bisa tambahin.

\section{Material and Related Works}

\section{Methodology}

\printbibliography

\end{document}
